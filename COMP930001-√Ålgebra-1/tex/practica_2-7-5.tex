% 16-04-22
% Práctica 2
% Ejercicio 7-v resuelto

\documentclass{article}
\usepackage[utf8]{inputenc}

\usepackage{enumerate}
\usepackage{enumitem}% http://ctan.org/pkg/enumitem
\usepackage{amsmath}
\usepackage{amssymb}
\usepackage{multicol}



\setlength\parindent{0pt}

\begin{document}

\begin{enumerate}[start=7]
  \item Probar que para todo $\it{n} \in \mathbb{N}$ se tiene
  
  \begin{enumerate}
    \begin{enumerate}[start=5, label=\roman*), itemsep=0.4ex, before={\everymath{\displaystyle}}]%
      \item $ \prod_{i=1}^{n}\frac{n+i  }{2i-3} = 2^{n}(1-2n) $ \label{eq-1}
    \end{enumerate}
    
\end{enumerate}
  
\end{enumerate}
\vspace{10.0}
Primero, establecemos $p(n)$
\begin{align*}
    p(n): \prod_{i=1}^{n}\frac{n+i}{2i-3} = 2^{n}(1-2n)
\end{align*}

Luego, probamos $p(1)$

\begin{alignat*}{2}
       p(1):& \hspace{20}\prod_{i=1}^{1}\frac{1+1}{2i-3} &&= 2^{1}\cdot(1-2\cdot1) \\
       p(1):& \hspace{30}\frac{2}{2\cdot1-3} &&= 2\cdot(-1) \\
       p(1):& \hspace{50}-2 &&= -2
\end{alignat*}

Por ende, $p(1)$ es verdadero. Ahora probemos el paso inductivo.
\\
\\
$\overbrace{p(n)}^{\it{Verdad \hspace{1} por \hspace{1} H.I.}} \hspace{-15} \Rightarrow p(n+1): $

\begin{alignat*}{2}
    p(n+1):& \prod_{i=1}^{n+1}\frac{n+1+i}{2i-3} &&= 2^{n+1}(1-2(n+1)) \\
    p(n+1):& \prod_{i=1}^{n+1}\frac{n+1+i}{2i-3} &&= n\cdot2^{n+2}+2^{n+1}
\end{alignat*}

Habiendo expandido completamente el término de la derecha, procedamos ahora a reescribir el término de la izquierda.

\begin{align*}
    \prod_{i=1}^{n+1}\frac{n+1+i}{2i-3} = \frac{\prod_{i=1}^{n+1}n+1+i}{\prod_{i=1}^{n+1}2i-3}
\end{align*}

\newpage

Luego, reescribimos los términos del numerador y el denominador respectivamente. 
\\\\
Primero, el denominador:
\begin{align*}
      &\prod_{i=1}^{n+1}2i-3 =            \\
    = &\prod_{i=1}^{n}2i-3 \cdot 2(n+1)-3 \\ 
    = &\prod_{i=1}^{n}2i-3 \cdot 2n-1
\end{align*}

Luego, para el numerador, hacemos un cambio de variable.\\Tal que $i+1 = j \Rightarrow j = 2$

\begin{align*}
      &\prod_{i=1}^{n+1}n+1+i =                                           \\
    = &\prod_{j=2}^{n+2}n+j                                               \\ 
    = &\prod_{j=1}^{n}n+j \cdot (n+n+1) \cdot (n+n+2) \cdot \frac{1}{n+1} \\
    = &\prod_{j=1}^{n}n+j \cdot (2n+1) \cdot (2n+2) \cdot \frac{1}{n+1}
\end{align*}

Nótese que para que la igualdad se cumpla, añadimos dos términos ($2n+1$ y $2n+2$) multiplicando al productorio original. Pero como además el nuevo productorio está desplazado una unidad en $j$, nos sobra el término inicial de $\prod_{j=1}^{n} n+j$ por lo que se lo quitamos, dividiendo al mismo por dicho término ($n+1$).
\\\\
Ahora, como $\prod_{j=1}^{n} n+j = \prod_{i=1}^{n} n+i+1$ es nuestra hipótesis inductiva, podemos reemplazar tal que:

\begin{align*}
     \frac{\prod_{i=1}^{n+1}n+1+i}{\prod_{i=1}^{n+1}2i-3} =
     \underbrace{\frac{\prod_{i=1}^{n}n+1+i}{\prod_{i=1}^{n}2i-3}}_{\textrm{Hipótesis}}
     \cdot (2n+1) \cdot (2n+2) \cdot \frac{1}{n+1} \cdot \frac{1}{2n-1}
\end{align*}

Luego,

\begin{align*}
     \therefore \frac{\prod_{i=1}^{n+1}n+1+i}{\prod_{i=1}^{n+1}2i-3} = 2^{n}(1-2n)
     \cdot (2n+1) \cdot (2n+2) \cdot \frac{1}{n+1} \cdot \frac{1}{2n-1}
\end{align*}

\end{document}
